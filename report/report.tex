\documentclass[draft,11pt,openright,monochrome,british,a4paper]{scrartcl}

%% PACKAGE LOADING
\usepackage[footnote,nomargin,index]{fixme}
\usepackage[pdftex,
            a4paper,
            pdftitle={Taming the Beast: A Report on Industry-University Research Ethics},
            pdfauthor={Willard Rafnsson, Szhau Lai, Sadegh Seddighi Khavidak},
            pdfsubject={Research Ethics},
            pdfkeywords={Ethics, Industry, University, Patent, Intellectual Property},
            bookmarksopen=false]{hyperref} % links etc. i bib        

%% NEW COMMANDS
\newcommand{\willard}[1]{\fixme{#1\textbf{\ --- Willard:}}}

%% PREP

%% BEGIN DOCUMENT

\begin{document}
%% TITLE, HEAD, FORMATTING, ETC.
%\title{Taming the Beast: A Case Analysis in Industry-University Research Ethics}
\title{Taming the Beast}
%\subtitle{Project by Group 7}
\subtitle{A Report on Industry-University Research Ethics}
\author{Willard Rafnsson \and Szhau Lai \and Sadegh Seddighi Khavidak}
\date{March 15, 2010}
\subject{Ethics, Science \& Society}
\maketitle

%% CONTENT GOES BELOW THIS LINE
\section{Introduction}
In the red corner we have the \emph{University}. Its role in society
is to be a place of unbiased inquiry, basic research, and critical
thinking~\cite{washburn2001}, advancing and improving the quality of
life through dissemination of knowledge. In the blue corner we have
the \emph{Industry}. All its activity is driven by a single purpose:
Profit~\cite{pain2008}. What is at stake is the integrity of the
university. And the university is loosing.

How does the conflict arise? To stay competitive, the industry wants
new ideas to the market. To that end, the industry channels funds to
university research. By introducing innovation into everyday life in
society, and by financing research, the industry is thus contributing
to advancement and life improvement. However, while the industry
sometimes gives grant money to individuals and projects with few or
``no strings'' attached, there usually is ``a catch'' set by the
industry to protect their interests~\cite{pain2008}. These
include~\cite{pain2008,washburn2001,resnik2003}: i) giving
intellectual property (IP) rights of discoveries\footnote{Sometimes
  discoveries not even funded by the company in
  question~\cite{washburn2001}} to the industy ii) delaying
publication by months or even years\footnote{Steven Rosenberg, a
  cancer researcher, is often asked to sign agreements to keep results
  of his research secret for up to 10 years, just to get access to
  reagents for his work.\cite{washburn2001}}, iii) right to review a
publication for confidential or patentable data, iv) forbidding or
limiting the scientist from sharing his discoveries or extending on
his work, v) requiring seats on university board, and vi) influencing
education curriculum. There are also examples of industry blatantly
disregarding ethics and academic conduct for profit reasons, including
i) doctoring an experiment or its result to yield results more
profitable for the company, ii) silencing results which are bad for
the company's business\footnote{Example: Morris silences two
  scientists who discover that i) Nicotine is addictive but bad for
  bloodvessels, and ii) there is a different substance that is more
  addivtive and less harmful to bloodvessels.}, and iii) sueing and
otherwise sabortage the carreers of scientists who act in society's
best interest\footnote{Example: Boots sues and discredits the work of
  one of its drug researchers who discovers and publishes that a Boots
  drug is less effective than mainstream drugs. USA would save
  millions by only using the mainstream drug~\cite{resnik2003}.}.

While universities could simply refuse these lucrative offers from
industry, in times of need when public funding is low, the deep
pockets of the private sector become alluring.

\section{Lessons Learned}
The industry desires innovations from the university, and the
university is attracted to the industry's money. But as we have seen,
\emph{if the reins are let go}, the beastly behaviour of the industry
will compromise the integrity of the university, which plays a vital
role in society.

The general problem is that i) industry adds more ``catch''es if it
can get away with it, and ii) there is no globally-accepted standard
for industry-university cooperation. \cite{washburn2001} has details
on i) and ii) \emph{on which we will extend and elaborate}.

As mentioned in~\cite{pain2008}, one of the most quarrelsome issues in
Industry-University cooperation is IP. In fact, it alone is usually
the reason behind pt. i)-iv) of ``the catch''es in receiving funding
from industry\footnote{While the fact that universities optain IP of
  their researchers' discoveries opens up a world of problems on its
  own, those are beyond the scope of this report. We do note, though,
  that in the USA, all but 12 schools are in fact not profiting from
  obtaining IP of their researchers' ideas~\cite{washburn2006}}. While
the purpose of IP law is supposedly to promote development of
knowledge~\cite{alger2001}, IP seems to be having the opposite
effect. In particular, if industry owns the IP of work done by
scientists, given the lengths the industry is willing to go to protect
its IP (we will see an example of this in the next section),
scientists do not feel they can talk freely to colleagues, in fear
that one will expose a corporate secret~\cite{washburn2001}.

We personally feel IP is a bad idea as it is so prone to abuse by
industry\footnote{There exist companies which do nothing but obtain IP
  rights and sue supposed infringers; see~\cite{sco}}. However, we do
not see a substitute for it, and since it seems to be here to stay, we
propose
\begin{enumerate}
\item[1.] Industry can only demand a \emph{patent}\footnote{A patent
    is a type of IP: A set of exclusive rights granted by a state to
    an inventor or their assignee for a limited period of time in
    exchange for a public disclosure of an invention.} on discoveries
  (strictly) within the scope of the funded project, only for the
  duration of the project, and only if the industry promises to profit
  from the discovery.
\end{enumerate}
This allows researchers to discuss discoveries openly and with
coworkers, thus not hampering dissemination of knowledge, while still
giving the industry a competitive edge. The patents must be very
specific, though, as not to give the industry more ammunition for
``patent-trolling''\footnote{Patent troll is a person/company
  enforcing its patents against one or more alleged infringers in a
  manner considered unduly aggressive or opportunistic, often with no
  intention to manufacture or market the patented invention.}. The
last part is necessary to ensure that the industry does not ``sit on''
the discovery (if the funding company is uninterested in profiting
from the discovery, another company might push the idea to market,
thus benefiting society).

It should be noted that a) the patent duration should vary from field
to field, and b) patenting does not make sense in all fields. In a), a
10 year patent may be reasonable in structural engineering, but is
nearly an eternity in computation science (that discipline is only 70
years old). In b), assuming mathematical truths cannot be patented,
then since programs are proofs (by the Curry-Howard
Isomorphism~\cite{curryhoward}), which are mathematical truths,
programs are inherently unpatentable. Yet, like most of our DNA, and
even some numbers, they are copyrighted.

Industry, while mostly motivated by profit, often get good ideas of
research directions. It can therefore be in the university's, and
society's, best interest to allow the industry to take some active
part in what goes on in the university. However, great care must be
taken here to control the beast. To make sure that a university does
not become the tool of a single industry or company, we propose:
\begin{enumerate}
\item[2.] Research should always be independent of individual company
  needs.
  \begin{enumerate}
  \item[2a.] Academic division and project funding, if received from
    industry, should come from multiple independent industries,
  \item[2b.] Industry representatives, if present in the university
    board, should be in minority, and from multiple independent industries.
  \end{enumerate}
\end{enumerate}
This diversity would hopefully ensure that i) research in universities
stays basic, and won't lean towards the practical requirements of a
particular industrial partner, ii) industry can give its input in
board meetings, and become aware of what occurs in the university,
without controlling it. At last,
\begin{enumerate}
\item[3.] Universities should not tailor student curriculum to
  industry needs.
\end{enumerate}
Again, the university's duty is to society, not to the industry. While
the industry can discover topics that should be introduced into
student curriculum, it should be kept in mind that the goal of
scientists is to serve society; not generate profit for industry.

We need to plant our foot down. It is ultimately the industry that
comes to us for creative aid. We must standardise how universities
interact with industry to maintain the integrity of the university,
lest the beast runs amok and demolishes our ivory towers.

\section{Case Study}
Here we analyse the case of Petr Taborsky, master student at the
University of South Florida, who was imprisoned for ``stealing his own
ideas'' from his university and the company which funded a project he
once worked on. See~\cite{jaroff1997} for details.
\begin{description}
\item[Stakeholders:]Petr Taborsky, versus beast: University/Professor and Florida Progress.
\item[Facts:]mostly in chronologial order:
  \begin{enumerate}
    \item Taborsky was a student.
    \item Taborsky was also an employee.\label{facts:employee}
    \item Taborsky was funded by project grant from "Florida Progress" to professor.\label{facts:funding}
    \item During which time no results turned up.
    \item Professor told Taborsky he could do with his ideas as he pleased.\label{facts:permission}
    \item Taborsky was then funded by other budgets.
    \item During which time Taborsky made a discovery, \emph{in his spare time}.
    \item Taborsky informs beast about discovery at meeting.\label{fact:inform}
    \item Florida Progress lays claim (Discovery worth millions).
    \item Taborsky refuses, takes his notes home.\label{fact:refuses}
    \item Beast sues and wins; obtains notes.
    \item Taborsky files and obtains patent.
    \item Beast threatens to sue if Taborsky does not relinquish patent.
    \item Taborsky refuses.
    \item Beast sues and wins; Taborsky imprisoned.\label{facts:imprisoned}
    \item Media becomes aware.
    \item Governor offers Taborsky clemency.
    \item Taborsky refuses.
    \end{enumerate}
  \item[Uncertainties:]key ones listed as follows:
    \begin{itemize}
    \item In~\ref{facts:permission}, how did Taborsky obtain
      permission? (on paper, word-of-mouth, etc).
    \item In~\ref{facts:employee}, the working contract (grounds for beast's claims) is dubious.
    \item In~\ref{facts:funding}, the relation of professor to
      company and the university's patent policy is not clear.
    \item In~\ref{facts:imprisoned}, did the beast obtain the patent?
    \end{itemize}
  \item[Missing Information:]we noted the following:
    \begin{itemize}
    \item Witnesses in~\ref{facts:permission},
    \item Novelty of Taborsky's idea,
    \item Case made by accuser and defendant in trial.
    \end{itemize}
  \item[Points of Conflict:]Taborsky's profit \& search for truth vs. Beast's profit.
  \item[Ethical Issues:]These two came to mind:
    \begin{itemize}
    \item Trust-based word-of-mouth contract is insufficient in the
      university environment. One would think it was enough.
    \item University more intested in profit than scientific progress,
      or wellfare of its student.
    \item Taborsky's obligations to the beast: Taborsky did use the
      beast's facilities to make his discoveries. Beast has grounds
      for \emph{some} claim.
    \end{itemize}
\end{description}

\subsubsection*{Options and Consequences:}
Unfortunately there are many uncertainties and much missing
information, which, if left variable, create an explosion of options
and consequences. Also, the options depend on where in the sequence of
events the choice is made. And just to add even more variance:
\cite{alger2001} explains who has the IP rights of a discovery you
make while working some someone (work-for-hire doctrine):
\begin{quote}
  Employers own the intellectual property rights to works created by
  the employees acting within the scope of their employment.
\end{quote}
We observe, however, that there is ``a gray area'' here: What is the
scope of employment? How close is the idea to what the employee is
paid to do? Also, the ``work-for-hire'' doctrine only applies if you
did the work as an employee. A student who is also an employee can
thus argue that he did the patentable work as a student, and perhaps
dodge the ``work-for-hire'' doctrine altogether.

We start with an option 0:
\begin{itemize}
\item \textbf{Option 0:} Before~\ref{fact:inform}, Taborsky could get
  the professor's promise that he can do what he wishes with his work
  on paper. \emph{Consequence:} If Taborsky succeeds, he might own his
  idea (unless something about the employment contract makes the
  professor's written promise void). However, this is likely to arouse
  suspicion in the professor.
\end{itemize}
While an interesting option, we feel that the most interesting
decision point is \ref{fact:refuses}. This is where Taborsky learns
about the value of his discovery, and where the decision Taborsky
makes will dramatically affect his future (as we have seen).

We will pick three options due to space constraints. These are:
\begin{itemize}
\item \textbf{Option 1:} Taborsky could optain his contract and IP
  obligations from the university (without talking to his professor,
  to avoid arousing suspicion), and when threatened by the beast,
  present the case to a lawyer, or the ``ethics police''. Taborsky
  will likely end up in a web of lengthy trials and court hearings,
  but if he manages to build a strong case (depends on his contract;
  if the beast ``owns Taborsky's brains'', there is not much Taborsky
  can do), he should win, despite how much money the beast throws into
  the trial.

\item \textbf{Option 2:} Taborsky could flee to his home country, The
  Czech Republic, make his government aware that he might become an
  outlaw in USA, and market his idea from home.

\item \textbf{Option 3:} Taborsky could negotiate with the beast. They
  did offer him a staff job. If he informs them that the discovery was
  made outside his contract, yet within their lab, and as a follow-up
  of contracted work, he could accept a staff job and a profit
  percentage.
 \end{itemize}
 
\subsubsection*{Assessment \& Action}
\begin{itemize}
\item \textbf{Option 2:} While Taborsky would prosper in the Czech
  Republic, he would never be able to visit USA again. His wife might
  not even be able to follow. This decision is a little unfair to the
  beast, as this discovery was made using their facilities. Finally,
  this decision will only allow the beast to oppress other students in
  the future.

\item \textbf{Option 3:} Knowing what industry is capable of, this is
  also a fairly safe option, which the beast would indeed be happy
  with as well. However, he would be working for a company that just
  cheated him out of millions. Who knows what kind of working
  environment he will be in.

\item \textbf{Option 1:} Perhaps a dangerous option, Taborsky could
  offer a fraction of his earnings to the beast for the use of his
  facilities. The beast will not be very pleased, but it did had
  little claim to the discovery in the first place. It would be good
  if any scientist made this decision, as the industry should not be
  allowed to get away with this kind of ``theft/bullying''.
 \end{itemize}

We choose: Option 1.

%% NO MORE CONTENT BELOW THIS LINE
\enlargethispage{30px}
\small
\bibliographystyle{alphaurl}
\bibliography{bibliography}

\end{document}









%% The new report from Sunday 14-March-2010


\documentclass[draft,11pt,openright,monochrome,british,a4paper]{scrartcl}

%% PACKAGE LOADING
\usepackage[footnote,nomargin,index]{fixme}

%% NEW COMMANDS
\newcommand{\willard}[1]{\fixme{#1\textbf{\ --- Willard:}}}

%% BEGIN DOCUMENT

\begin{document}
%% TITLE, HEAD, FORMATTING, ETC.
\title{Taming the Beast: A Case Analysis in Industry-University Research Ethics}
\author{Willard Rafnsson \and Szhau Lai \and Sadegh Seddighi Khavidak}
\subtitle{Project by Group 7}
\date{March 15, 2010}
\subject{Ethics, Science \& Society}
\maketitle

%% CONTENT GOES BELOW THIS LINE
  \section{Lessons learned}
  Universities from beginning aim for advancing society and improve quality of life by dissemination of knowledge. So to say universities are Institutes for unbiased critical thinking that opens the doors for the break through, inventions and explorations. On the other hand industries are business enterprizes that mostly aim for profit. However, as the business world became more competitive, having new ideas in the market would be a great privilege for an industry. SO the industries are more and more funding especial programs in the universities in order to exploit the academic merit in creativity. This collaboration helps both parts to grow and flourish. There are many academic innovations funded by industries that improved the quality of life of the people. Moreover, the academic results might be sometimes too theoretical and industry is helping to make more applicable in real life. However, it is not all that nice. There are some side effects on this close cooperation.
  
  Industries, as the funder of project in the universities can have bad effects in several ways.
  \begin{itemize}
  \item Industry tailors test results to their favor.
  \item Industry blocks publications which discredit their work / hurt their industry.
  \item Industry influences education.
  \end{itemize}
  
  Industry wants the universities innovations. Universities want the industries money. \textbf{If
    reins are let go}, industry assimilates universities. But why this assimilation should not happen? The direct consequences
  would be that university becomes a business and people loose trust in it,\ldots{}). How much do we let the universities get away with? How do
  we prevent assimilation?
  
  The following items are the proposed measures in order to keep the integrity of the academy and independence.
    \begin{itemize}
  \item The academic divisions and projects should be funded from multiple independent partners instead of one or two financer.
  \item If university wants to invite industry fermentative into the university board or any other decision making body, the industry part of the board should be always in minority 
  \item Having just one industry in the decision making bodies should be avoided.
  \item There should be an explicit regulation saying university research should be independent of individual company needs.
  \item A collective guidelines for cooperation with industry must be made from the ministries of science that prevents the industries manipulating the academy with their contracts.
  \item The intellectual property rights should be clearly set for the university-academy cooperations.
  \end{itemize}
  
  The intellectual property that has been mentioned as the last item, became the critical issue over the last few years. As the university projects being funded by a certain industrial partner, the results of those projects are not published or utilized according to the academic tradition. For example, in IT world, if there is an innovation, it will become obsolete if it is not published in 6 months. So if there is a pressure from the industrial partner to stop the publication for half a year, it implies that the university forgets its duty for "being on the edge of knowledge and letting the world benefit from it". A recent study in the Journal of American Medical Association found that 20 percent of scientist delay the publication of their work for more than six months in order to protect proprietary information.
  
  Also when it comes to the innovation rights there is a great clash of interests between the researcher in the universities and the industrial partners. Although, it is the researcher that works and invents, the university claims the whole right of the patents and now, industry is claiming its share as well. So there should be a clear definition of how and where the rights would go.
  
  
\newpage


\section{Case study : The Master Student at University of South Florida}

\subsection*{Stakeholders: Petr Taborsky, University/Professor, Florida Progress}

\subsection*{Facts}


\begin{itemize}
\item   Project: Remove ammonia from waste water.
\item	Taborsky continued working on company-sponsored research project after it was terminated, and he got the permission from professor (word of mouth).
\item	He god a good result and was told that his idea worth million. The professor and the company request him to deliver his discovery. Taborsky refused.
\item	Taborsky applied for patent and the University threatened him to hand over the patent to USF or put him in jail. Taborsky refused and was sent to chain gang.
\item	Professor applied for another patent after Taborsky but failed to provide experiment data. Either to reassure the company or give support to the patent application, the university wants Taborsky in jail by every means.
\item	Media gets aware of this and Governor offered Taborsky a clemency, which Taborsky refused.

 \end{itemize}
 
\subsection*{Uncertainties:}
How did Taborsky receive the permission is unclear. The working contract (proof) is dubious. The relation of professor to company and the university's patent policy is not clear.

\subsection*{Missing information:}
Witnesses and the novalty of Taborsky's idea.

\subsection*{Points of conflict:}
Taborsky's idea and profit vs University and company's profit.

\subsection*{Ethical issues:}
Trust-based word-of-mount contract. (one would think that this would be sufficient in a university. But it is not). For some reason, university is more interested in money that scientific progress.

Taborsky's ethical obligation to University/Professor:
The University/Professor deserve piece of the pie for the facility Taborsky has used in the lab, even the idea is his own.

\subsection*{Options and consequences :}
\begin{itemize}
\item Option1: Taborsky's option
Taborsky talk to Florida Progress representative and his professor first but end up as unsatisfied result. He could consult the university patent office to check the IP policy. If the policy is not fair to him, he could postpone it after he graduated.
Consequence1 : He could face the risk that someone is doing the same thing and apply patent before he has the opportunity. The company and university can still sue him, but they have very little proof to win the trial.

\item Option 2: To essentially solve the problem is to change the system and the policy. It's clearly the Florida progress has significant influence in USF, such that the professor and USF board need to flatter the company.
That leads to our case 1 solution, the university should get funding from multiple sources instead of keeping close relation with one.
Consequence2 : In every perspective, this is the best way. Except the Florida progress won't be satisfied.
 \end{itemize}
 
\subsection*{Assessment & Action}
\begin{itemize}
\item By all means of test, publicity, reversibility, etc. the action cannot satisfy everyone except the individual. However, there is no fair policy, why should people follow.
\item Option2
The university is a public resources, it should keep it supreme dominate position in front of the industrial partner. Option2 solve the problem for future, but it's difficult to implement it.
 \end{itemize}

%% NO MORE CONTENT BELOW THIS LINE

\bibliographystyle{plainnat}
\bibliography{bibliography}

\end{document}
