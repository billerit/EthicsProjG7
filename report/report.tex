\documentclass[draft,11pt,openright,monochrome,british,a4paper]{scrartcl}

%% PACKAGE LOADING
\usepackage[footnote,nomargin,index]{fixme}

%% NEW COMMANDS
\newcommand{\willard}[1]{\fixme{#1\textbf{\ --- Willard:}}}

%% BEGIN DOCUMENT

\begin{document}
%% TITLE, HEAD, FORMATTING, ETC.
%\title{Taming the Beast: A Case Analysis in Industry-University Research Ethics}
\title{Taming the Beast}
%\subtitle{Project by Group 7}
\subtitle{A Case Analysis in Industry-University Research Ethics}
\author{Willard Rafnsson \and Szhau Lai \and Sadegh Seddighi Khavidak}
\date{March 15, 2010}
\subject{Ethics, Science \& Society}
\maketitle

%% CONTENT GOES BELOW THIS LINE
\section{Introduction}
In the red corner we have the \emph{University}. Its role in society
is to be a place of unbiased inquiry, basic research, and critical
thinking\cite{washburn2001}, advancing and improving the quality of
life through dissemination of knowledge. In the blue corner we have
the \emph{Industry}. All its activity is driven by a single purpose:
Profit~\cite{pain2008}. What is at stake is the integrity of the
university. And the university is loosing.

How does the conflict arise? To stay competitive, the industry wants
new ideas to the market. To that end, the industry channels funds to
university research. By introducing innovation into everyday life in
society, and by financing research, the industry is thus contributing
to advancement and life improvement. However, while the industry
sometimes gives grant money to individuals and projects with few or
``no strings'' attached, there usually is ``a catch'' set by the
industry to protect their interests~\cite{pain2008}. These
include~\cite{pain2008,washburn2001,resnik2003}: i) giving
intellectual property (IP) rights of discoveries\footnote{Sometimes
  discoveries not even funded by the company in
  question~\cite{washburn2001}} to the industy ii) delaying
publication by months or even years\footnote{Steven Rosenberg, a
  cancer researcher, is often asked to sign agreements to keep results
  of his research secret for up to 10 years, just to get access to
  reagents for his work.\cite{washburn2001}}, iii) right to review a
publication for confidential or patentable data, iv)
forbidding or limiting the scientist from sharing his discoveries or
extending on his work, v) requiring seats on university board, and vi)
influencing education curriculum. There are also examples of industry
blatantly disregarding academic conduct for profit reasons, including
i) doctoring an experiment or its result to yield results more
profitable for the company, ii) silencing results which are bad for
the company's business\footnote{Example: Morris silences two
  scientists who discover that i) Nicotine is addictive but bad for
  bloodvessels, and ii) there is a different substance that is more
  addivtive and less harmful to bloodvessels.}, and iii) sueing and
otherwise sabortage the carreers of scientists who act in society's
best interest\footnote{Example: Boots sues and discredits the work of
  one of its drug researchers who discovers and publishes that a Boots
  drug is less effective than mainstream drugs. USA would save
  millions by only using the mainstream drug~\cite{resnik2003}.}.

While universities could simply refuse these lucrative offers from
industry, in times of need when public funding is low, the deep
pockets of the private sector become alluring.

\section{Lessons Learned}
The industry desires innovations from the university, and the
university is attracted to the industry's money. But as we have seen,
\emph{if the reins are let go}, the beastly behaviour of the industry
will compromise the integrity of the university, which plays a vital
role in society.

An interesting observation made by Washburn on whether patenting and
licenting research pays off for universities is~\cite{washburn2006}:
\begin{quote}
  No. [\ldots{}] The overwhelming majority of the [\$1.3 billion]
  profits go to less than two dozen schools. And the rest of the
  schools are barely breaking even or losing money on these commercial
  activities.
\end{quote}


do they *really* need to impose all these constraints?

\willard{We need to include mini-summaries of cases and examples
  supporting the claims we make here. We also need to elaborate more
  on ``lessons learned''. Most importantly, we must explain how IP
  plays a key role here!}\\

protect their interests (profit). Willing to go to great, unethical, lengths.
\begin{itemize}
\item How much do we let the universities get away with? How do we
  prevent assimilation?
  \begin{itemize}
  \item Make sure to be funded from multiple independent
    partners.\willard{Note that it can be a good idea to have
      \emph{some} industry members on the board, namely due to the
      benefits to society industry can bring.}
  \item If you must invite industry into the university board, make
    sure they are a minority, and no single industry dominates.
  \item Laws on the following should be past: University research should be independent of individual
    company needs (make law)
  \item A collective guidelines for cooperation with industry must
    be made.
  \item See \cite{pain2008}, section ``strings''.
  \end{itemize}
\end{itemize}


We need to plant our foot down. It is ultimately the industry that
comes to us for creative aid. We must standardise how universities
interact with industry to maintain the integrity of the university,
lest the beast runs amok and demolishes our ivory towers.

\section{Case study : The Master Student at University of South Florida}

\subsection*{Stakeholders: Petr Taborsky, University/Professor, Florida Progress}

\subsection*{Facts}


\begin{itemize}
\item   Project: Remove ammonia from waste water.
\item	Taborsky continued working on company-sponsored research project after it was terminated, and he got the permission from professor (word of mouth).
\item	He god a good result and was told that his idea worth million. The professor and the company request him to deliver his discovery. Taborsky refused.
\item	Taborsky applied for patent and the University threatened him to hand over the patent to USF or put him in jail. Taborsky refused and was sent to chain gang.
\item	Professor applied for another patent after Taborsky but failed to provide experiment data. Either to reassure the company or give support to the patent application, the university wants Taborsky in jail by every means.
\item	Media gets aware of this and Governor offered Taborsky a clemency, which Taborsky refused.

 \end{itemize}
 
\subsection*{Uncertainties:}
How did Taborsky receive the permission is unclear. The working contract (proof) is dubious. The relation of professor to company and the university's patent policy is not clear.

\subsection*{Missing information:}
Witnesses and the novalty of Taborsky's idea.

\subsection*{Points of conflict:}
Taborsky's idea and profit vs University and company's profit.

\subsection*{Ethical issues:}
Trust-based word-of-mount contract. (one would think that this would be sufficient in a university. But it is not). For some reason, university is more interested in money that scientific progress.

Taborsky's ethical obligation to University/Professor:
The University/Professor deserve piece of the pie for the facility Taborsky has used in the lab, even the idea is his own.

\subsection*{Options and consequences :}
\begin{itemize}
\item Option1: Taborsky's option
Taborsky talk to Florida Progress representative and his professor first but end up as unsatisfied result. He could consult the university patent office to check the IP policy. If the policy is not fair to him, he could postpone it after he graduated.
Consequence1 : He could face the risk that someone is doing the same thing and apply patent before he has the opportunity. The company and university can still sue him, but they have very little proof to win the trial.

\item Option 2: To essentially solve the problem is to change the system and the policy. It's clearly the Florida progress has significant influence in USF, such that the professor and USF board need to flatter the company.
That leads to our case 1 solution, the university should get funding from multiple sources instead of keeping close relation with one.
Consequence2 : In every perspective, this is the best way. Except the Florida progress won't be satisfied.
 \end{itemize}
 
\subsection*{Assessment \& Action}
\begin{itemize}
\item By all means of test, publicity, reversibility, etc. the action cannot satisfy everyone except the individual. However, there is no fair policy, why should people follow.
\item Option2
The university is a public resources, it should keep it supreme dominate position in front of the industrial partner. Option2 solve the problem for future, but it's difficult to implement it.
 \end{itemize}

%% NO MORE CONTENT BELOW THIS LINE

\bibliographystyle{alpha}
\bibliography{bibliography}

\end{document}









%% The new report from Sunday 14-March-2010


\documentclass[draft,11pt,openright,monochrome,british,a4paper]{scrartcl}

%% PACKAGE LOADING
\usepackage[footnote,nomargin,index]{fixme}

%% NEW COMMANDS
\newcommand{\willard}[1]{\fixme{#1\textbf{\ --- Willard:}}}

%% BEGIN DOCUMENT

\begin{document}
%% TITLE, HEAD, FORMATTING, ETC.
\title{Taming the Beast: A Case Analysis in Industry-University Research Ethics}
\author{Willard Rafnsson \and Szhau Lai \and Sadegh Seddighi Khavidak}
\subtitle{Project by Group 7}
\date{March 15, 2010}
\subject{Ethics, Science \& Society}
\maketitle

%% CONTENT GOES BELOW THIS LINE
  \section{Lessons learned}
  Universities from beginning aim for advancing society and improve quality of life by dissemination of knowledge. So to say universities are Institutes for unbiased critical thinking that opens the doors for the break through, inventions and explorations. On the other hand industries are business enterprizes that mostly aim for profit. However, as the business world became more competitive, having new ideas in the market would be a great privilege for an industry. SO the industries are more and more funding especial programs in the universities in order to exploit the academic merit in creativity. This collaboration helps both parts to grow and flourish. There are many academic innovations funded by industries that improved the quality of life of the people. Moreover, the academic results might be sometimes too theoretical and industry is helping to make more applicable in real life. However, it is not all that nice. There are some side effects on this close cooperation.
  
  Industries, as the funder of project in the universities can have bad effects in several ways.
  \begin{itemize}
  \item Industry tailors test results to their favor.
  \item Industry blocks publications which discredit their work / hurt their industry.
  \item Industry influences education.
  \end{itemize}
  
  Industry wants the universities innovations. Universities want the industries money. \textbf{If
    reins are let go}, industry assimilates universities. But why this assimilation should not happen? The direct consequences
  would be that university becomes a business and people loose trust in it,\ldots{}). How much do we let the universities get away with? How do
  we prevent assimilation?
  
  The following items are the proposed measures in order to keep the integrity of the academy and independence.
    \begin{itemize}
  \item The academic divisions and projects should be funded from multiple independent partners instead of one or two financer.
  \item If university wants to invite industry fermentative into the university board or any other decision making body, the industry part of the board should be always in minority 
  \item Having just one industry in the decision making bodies should be avoided.
  \item There should be an explicit regulation saying university research should be independent of individual company needs.
  \item A collective guidelines for cooperation with industry must be made from the ministries of science that prevents the industries manipulating the academy with their contracts.
  \item The intellectual property rights should be clearly set for the university-academy cooperations.
  \end{itemize}
  
  The intellectual property that has been mentioned as the last item, became the critical issue over the last few years. As the university projects being funded by a certain industrial partner, the results of those projects are not published or utilized according to the academic tradition. For example, in IT world, if there is an innovation, it will become obsolete if it is not published in 6 months. So if there is a pressure from the industrial partner to stop the publication for half a year, it implies that the university forgets its duty for "being on the edge of knowledge and letting the world benefit from it". A recent study in the Journal of American Medical Association found that 20 percent of scientist delay the publication of their work for more than six months in order to protect proprietary information.
  
  Also when it comes to the innovation rights there is a great clash of interests between the researcher in the universities and the industrial partners. Although, it is the researcher that works and invents, the university claims the whole right of the patents and now, industry is claiming its share as well. So there should be a clear definition of how and where the rights would go.
  
  
\newpage


\section{Case study : The Master Student at University of South Florida}

\subsection*{Stakeholders: Petr Taborsky, University/Professor, Florida Progress}

\subsection*{Facts}


\begin{itemize}
\item   Project: Remove ammonia from waste water.
\item	Taborsky continued working on company-sponsored research project after it was terminated, and he got the permission from professor (word of mouth).
\item	He god a good result and was told that his idea worth million. The professor and the company request him to deliver his discovery. Taborsky refused.
\item	Taborsky applied for patent and the University threatened him to hand over the patent to USF or put him in jail. Taborsky refused and was sent to chain gang.
\item	Professor applied for another patent after Taborsky but failed to provide experiment data. Either to reassure the company or give support to the patent application, the university wants Taborsky in jail by every means.
\item	Media gets aware of this and Governor offered Taborsky a clemency, which Taborsky refused.

 \end{itemize}
 
\subsection*{Uncertainties:}
How did Taborsky receive the permission is unclear. The working contract (proof) is dubious. The relation of professor to company and the university's patent policy is not clear.

\subsection*{Missing information:}
Witnesses and the novalty of Taborsky's idea.

\subsection*{Points of conflict:}
Taborsky's idea and profit vs University and company's profit.

\subsection*{Ethical issues:}
Trust-based word-of-mount contract. (one would think that this would be sufficient in a university. But it is not). For some reason, university is more interested in money that scientific progress.

Taborsky's ethical obligation to University/Professor:
The University/Professor deserve piece of the pie for the facility Taborsky has used in the lab, even the idea is his own.

\subsection*{Options and consequences :}
\begin{itemize}
\item Option1: Taborsky's option
Taborsky talk to Florida Progress representative and his professor first but end up as unsatisfied result. He could consult the university patent office to check the IP policy. If the policy is not fair to him, he could postpone it after he graduated.
Consequence1 : He could face the risk that someone is doing the same thing and apply patent before he has the opportunity. The company and university can still sue him, but they have very little proof to win the trial.

\item Option 2: To essentially solve the problem is to change the system and the policy. It's clearly the Florida progress has significant influence in USF, such that the professor and USF board need to flatter the company.
That leads to our case 1 solution, the university should get funding from multiple sources instead of keeping close relation with one.
Consequence2 : In every perspective, this is the best way. Except the Florida progress won't be satisfied.
 \end{itemize}
 
\subsection*{Assessment & Action}
\begin{itemize}
\item By all means of test, publicity, reversibility, etc. the action cannot satisfy everyone except the individual. However, there is no fair policy, why should people follow.
\item Option2
The university is a public resources, it should keep it supreme dominate position in front of the industrial partner. Option2 solve the problem for future, but it's difficult to implement it.
 \end{itemize}

%% NO MORE CONTENT BELOW THIS LINE

\bibliographystyle{alpha}
\bibliography{bibliography}

\end{document}
